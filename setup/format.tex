% !Mode:: "TeX:UTF-8" 

\theoremstyle{plain}
\theorembodyfont{\song\rmfamily}
\theoremheaderfont{\hei\rmfamily}
\newtheorem{definition}{\hei 定义}[chapter]
\newtheorem{example}{\hei 例}[chapter]
\newtheorem{algo}{\hei 算法}[chapter]
\newtheorem{theorem}{\hei 定理}[chapter]
\newtheorem{axiom}{\hei 公理}[chapter]
\newtheorem{proposition}{\hei 命题}[chapter]
\newtheorem{lemma}{\hei 引理}[chapter]
\newtheorem{corollary}{\hei 推论}[chapter]
\newtheorem{remark}{\hei 注解}[chapter]
\newenvironment{proof}{\noindent{\hei 证明:}}{\hfill $ \square $ \vskip 4mm}
\theoremsymbol{$\square$}
\setlength{\theorempreskipamount}{0pt}
\setlength{\theorempostskipamount}{-2pt}

\allowdisplaybreaks[4]

%\CJKcaption {gb_452} 
%\CJKtilde
\setlength{\parindent}{2em}

\arraycolsep=1.6pt

\renewcommand\contentsname{\hei 目~~~~录}

\CTEXsetup[number={\arabic{chapter}}]{chapter}
\renewcommand\chaptername{第~\thechapter~章}

%\CTEXsetup[name={第,章}]{chapter}
\setcounter{secnumdepth}{4} \setcounter{tocdepth}{2}


\titleformat{\chapter}{\center\xiaoer\hei}{\chaptertitlename}{0.5em}{}
\titlespacing{\chapter}{0pt}{-5.5mm}{8mm}
\titleformat{\section}{\xiaosan\hei}{\thesection}{0.5em}{}
\titlespacing{\section}{0pt}{4.5mm}{4.5mm}
\titleformat{\subsection}{\sihao\hei}{\thesubsection}{0.5em}{}
\titlespacing{\subsection}{0pt}{4mm}{4mm}
\titleformat{\subsubsection}{\xiaosi\hei}{\thesubsubsection}{0.5em}{}
\titlespacing{\subsubsection}{0pt}{0pt}{0pt}

\titlecontents{chapter}[3.8em]{\hspace{-3.8em}\hei}{\thecontentslabel~~}{}{\titlerule*[4pt]{.}\contentspage}
\dottedcontents{section}[32pt]{}{21pt}{0.3pc}
\dottedcontents{subsection}[53pt]{}{30pt}{0.3pc}


% 按工大标准, 缩小目录中各级标题之间的缩进,使它们相隔一个字符距离,也就是12pt
\makeatletter
\renewcommand*\l@chapter{\@dottedtocline{0}{0em}{5em}}%控制英文目录: 细点\@dottedtocline  粗点\@dottedtoclinebold
\renewcommand*\l@section{\@dottedtocline{1}{1em}{1.8em}}
\renewcommand*\l@subsection{\@dottedtocline{2}{2em}{2.5em}}


% 定义页眉和页脚
\newcommand{\makeheadrule}{
\rule[7pt]{\textwidth}{0.75pt} \\[-23pt]
\rule{\textwidth}{2.25pt}}
\renewcommand{\headrule}{
    {\if@fancyplain\let\headrulewidth\plainheadrulewidth\fi
     \makeheadrule}}
\pagestyle{fancyplain}

%去掉章节标题中的数字
%%不要注销这一行,否则页眉会变成:“第1章1  绪论”样式
\renewcommand{\chaptermark}[1]{\markboth{\chaptertitlename~\ #1}{}}
\fancyhf{}

%在book文件类别下,\leftmark自动存录各章之章名,\rightmark记录节标题
%% 页眉字号 工大要求 小五
%根据单双面打印设置不同的页眉;

\ifxueweidoctor
  \fancyhead[CO]{\song \xiaowu\leftmark}
  \fancyhead[CE]{\song \xiaowu 哈尔滨工业大学\cxueke\cxuewei 学位论文}%
  \fancyfoot[C,C]{\xiaowu -~\thepage~-}
\else
  \fancyhead[CO]{\song \xiaowu 哈尔滨工业大学\cxueke\cxuewei 学位论文}
  \fancyhead[CE]{\song \xiaowu 哈尔滨工业大学\cxueke\cxuewei 学位论文}%
  \fancyfoot[C,C]{\xiaowu -~\thepage~-}
\fi

\renewcommand\frontmatter{\cleardoublepage
  \@mainmatterfalse
  \pagenumbering{Roman}}

% 调整罗列环境的布局
\setitemize{leftmargin=0em,itemsep=0em,partopsep=0em,parsep=0em,topsep=0em,itemindent=3em}
\setenumerate{leftmargin=0em,itemsep=0em,partopsep=0em,parsep=0em,topsep=0em,itemindent=3.5em}

\newcommand{\citeup}[1]{\textsuperscript{\cite{#1}}}

% 定制浮动图形和表格标题样式
\captionnamefont{\wuhao}
\captiontitlefont{\wuhao}
\captiondelim{~~}
%\captionstyle{\hang}
\hangcaption
\renewcommand{\subcapsize}{\wuhao}
\setlength{\abovecaptionskip}{0pt}
\setlength{\belowcaptionskip}{0pt}

% 自定义项目列表标签及格式 \begin{publist} 列表项 \end{publist}
\newcounter{pubctr} %自定义新计数器
\newenvironment{publist}{%%%%%定义新环境
\begin{list}{[\arabic{pubctr}]} %%标签格式
    {
     \usecounter{pubctr}
     \setlength{\leftmargin}{1.7em}     % 左边界 \leftmargin =\itemindent + \labelwidth + \labelsep
     \setlength{\itemindent}{0em}     % 标号缩进量
     \setlength{\labelsep}{0.5em}       % 标号和列表项之间的距离,默认0.5em
     \setlength{\rightmargin}{0em}    % 右边界
     \setlength{\topsep}{0ex}         % 列表到上下文的垂直距离
     \setlength{\parsep}{0ex}         % 段落间距
     \setlength{\itemsep}{0ex}        % 标签间距
     \setlength{\listparindent}{0pt} % 段落缩进量
    }}
{\end{list}}%%%%%

% 默认字体
\renewcommand\normalsize{
  \@setfontsize\normalsize{12pt}{12pt}
  \setlength\abovedisplayskip{4pt}
  \setlength\abovedisplayshortskip{4pt}
  \setlength\belowdisplayskip{\abovedisplayskip}
  \setlength\belowdisplayshortskip{\abovedisplayshortskip}
  \let\@listi\@listI}
  
% 设置行距和段落间垂直距离
\def\defaultfont{\renewcommand{\baselinestretch}{1.62}\normalsize\selectfont}
\renewcommand{\CJKglue}{\hskip 0.56pt plus 0.08\baselineskip} 
%加大字间距,使每行34个字,若要使得每行33个字,则将0.56pt替换为0.96pt。
\predisplaypenalty=0  %公式之前可以换页,公式出现在页面顶部

% 封面、摘要、版权、致谢格式定义
\def\ctitle#1{\def\@ctitle{#1}}\def\@ctitle{}
\def\cdegree#1{\def\@cdegree{#1}}\def\@cdegree{}
\def\caffil#1{\def\@caffil{#1}}\def\@caffil{}
\def\csubject#1{\def\@csubject{#1}}\def\@csubject{}
\def\cauthor#1{\def\@cauthor{#1}}\def\@cauthor{}
\def\csupervisor#1{\def\@csupervisor{#1}}\def\@csupervisor{}
\def\cassosupervisor#1{\def\@cassosupervisor{{\hei 副 \hfill 导 \hfill 师} & #1\\}}\def\@cassosupervisor{}
\def\ccosupervisor#1{\def\@ccosupervisor{{\hei 联 \hfill 合\hfill 导 \hfill 师} & #1\\}}\def\@ccosupervisor{}
\def\cdate#1{\def\@cdate{#1}}\def\@cdate{}
\long\def\cabstract#1{\long\def\@cabstract{#1}}\long\def\@cabstract{}
\def\ckeywords#1{\def\@ckeywords{#1}}\def\@ckeywords{}

\def\etitle#1{\def\@etitle{#1}}\def\@etitle{}
\def\edegree#1{\def\@edegree{#1}}\def\@edegree{}
\def\eaffil#1{\def\@eaffil{#1}}\def\@eaffil{}
\def\esubject#1{\def\@esubject{#1}}\def\@esubject{}
\def\eauthor#1{\def\@eauthor{#1}}\def\@eauthor{}
\def\esupervisor#1{\def\@esupervisor{#1}}\def\@esupervisor{}
\def\eassosupervisor#1{\def\@eassosupervisor{\textbf{Associate Supervisor:} & #1\\}}\def\@eassosupervisor{}
\def\ecosupervisor#1{\def\@ecosupervisor{\textbf{Co Supervisor:} & #1\\}}\def\@ecosupervisor{}
\def\edate#1{\def\@edate{#1}}\def\@edate{}
\long\def\eabstract#1{\long\def\@eabstract{#1}}\long\def\@eabstract{}
\long\def\NotationList#1{\long\def\@NotationList{#1}}\long\def\@NotationList{}
\def\ekeywords#1{\def\@ekeywords{#1}}\def\@ekeywords{}
\def\natclassifiedindex#1{\def\@natclassifiedindex{#1}}\def\@natclassifiedindex{}
\def\internatclassifiedindex#1{\def\@internatclassifiedindex{#1}}\def\@internatclassifiedindex{}
\def\statesecrets#1{\def\@statesecrets{#1}}\def\@statesecrets{}

% 定义封面
\def\makecover{
    \begin{titlepage}
    % 封面一
   \vspace*{0.8cm}
   \begin{center}
    \centerline{\xiaoyi\song\textbf{\cxuewei 学位论文}}

    \vspace{1cm}

    \parbox[t][2.8cm][t]{\textwidth}{
    \begin{center}\erhao\hei\@ctitle\end{center} }

    \parbox[t][5.1cm][t]{\textwidth}{ %英文标题太长时可以采用\xiaoer
    \begin{center}\erhao\textbf{\@etitle}\end{center} }

    \parbox[t][7.4cm][t]{\textwidth}{
    \begin{center}\xiaoer\song\textbf{\@cauthor}\end{center}}

    \parbox[t][1.4cm][t]{\textwidth}{
    \begin{center}\kaishu\xiaoer\textbf{哈尔滨工业大学}\end{center} }
    
    {\song\xiaoer\textbf{\@cdate}}

    \end{center}

    % 封二 空白页
    \ifxueweidoctor
      \newpage
      ~~~\vspace{1em}
      \thispagestyle{empty}
    \fi

    %内封
    \newpage
    \thispagestyle{empty}

\begin{center}

			{\song \xiaosi
			\begin{tabular}{@{}r@{:}l@{}}
			国内图书分类号 & \@natclassifiedindex\\
 			国际图书分类号 & \@internatclassifiedindex
			\end{tabular}}\hfill
			{\song \xiaosi
			\begin{tabular}{@{}r@{:}l@{}}
			学校代码 & 10213\\
 			密级 &  公开
			\end{tabular}}
    \parbox[t][3.2cm][t]{\textwidth}{\begin{center} \end{center} }

    \parbox[t][2.4cm][t]{\textwidth}{\xiaoer
    \begin{center} {\song \bfseries \@cdegree 学位论文 }\end{center} }

    \parbox[t][5cm][t]{\textwidth}{\erhao
    \begin{center} {\hei  \@ctitle}\end{center} }
	\parbox[t][9.8cm][b]{\textwidth}
     {\sihao
    \begin{center} \renewcommand{\arraystretch}{1.62} \song
    \begin{tabular}{l@{:}l}
    {\hei \xueweishort \hfill 士\hfill 研\hfill 究\hfill 生}           & \@cauthor\\
    {\hei 导\hfill 师}                       & \@csupervisor\\
	\@cassosupervisor
	\@ccosupervisor
    {\hei 申\hfill 请\hfill 学\hfill 位} & \@cdegree\\
    {\hei 学\hfill 科}           & \@csubject\\
    {\hei 所\hfill 在\hfill 单\hfill 位} & \@caffil\\
    {\hei 答\hfill 辩\hfill 日\hfill 期} & \@cdate\\
    {\hei 授予学位单位}                     & 哈尔滨工业大学
    \end{tabular} \renewcommand{\arraystretch}{1}
    \end{center} }
\end{center}

%%%%%%增加一空白页
  \ifxueweidoctor
    \newpage
    ~~~\vspace{1em}
    \thispagestyle{empty}
  \fi

    % 英文封面
    \newpage
    \thispagestyle{empty}

    {
    \xiaosi\noindent Classified Index: \@natclassifiedindex \\
                  U.D.C:  \@internatclassifiedindex }
    \begin{center}
    \parbox[t][1.6cm][t]{\textwidth}{\begin{center} \end{center} }
    \parbox[t][3.5cm][t]{\textwidth}{\xiaoer
    \begin{center} {  Dissertation for the {\exueweier} Degree in \exueke}\end{center} } %与中文保持一致,删除in {\exueke}

    \parbox[t][7cm][t]{\textwidth}{\erhao
    \begin{center} { \bfseries \@etitle}\end{center} }

%★★★★若信息内容不太长,不会引起信息内容分行时,使用tabular环境,否则使用下面的tabularx环境。
    {\sihao\renewcommand{\arraystretch}{1.3}
    \begin{tabular}{@{}l@{~}l@{}}
    \textbf{Candidate:}                     &  \@eauthor\\
    \textbf{Supervisor:}                    &  \@esupervisor\\
	  \@eassosupervisor
	  \@ecosupervisor
    \textbf{Academic Degree Applied for:}   &  \@edegree\\
    \textbf{Specialty:}                     &  \@esubject\\
    \textbf{Affiliation:}                   &  \@eaffil\\
    \textbf{Date of Defence:}               &  \@edate\\
    \textbf{Degree-Conferring-Institution:} &  Harbin Institute of Technology
    \end{tabular}\renewcommand{\arraystretch}{1}}

    %{\sihao\renewcommand{\arraystretch}{1.3}
    %\begin{tabularx}{\textwidth}{@{}l@{~}X@{}}
    %\textbf{Candidate:}                     &  \@eauthor\\
    %\textbf{Supervisor:}                    &  \@esupervisor\\
		%\@eassosupervisor
	  %\@ecosupervisor
    %\textbf{Academic Degree Applied for:}   &  \@edegree\\
    %\textbf{Specialty:}                     &  \@esubject\\
    %\textbf{Affiliation:}                   &  \@eaffil\\
    %\textbf{Date of Defence:}               &  \@edate\\
    %\textbf{Degree-Conferring-Institution:} &  Harbin Institute of Technology
    %\end{tabularx}\renewcommand{\arraystretch}{1}}

    \end{center}
    \end{titlepage}

%%%%%%增加一空白页
  \ifxueweidoctor
    \newpage
    ~~~\vspace{1em}
    \thispagestyle{empty}
  \fi
%%%%%%%%%%%%%%%%%%%   Abstract and keywords  %%%%%%%%%%%%%%%%%%%%%%%
\clearpage

\BiAppendixChapter{摘\quad 要}{Abstract (In Chinese)}

\setcounter{page}{1}
\song\defaultfont
\@cabstract
\vspace{\baselineskip}

\hangafter=1\hangindent=51pt\noindent
{\hei 关键词}:\@ckeywords

%%%%%%%%%%%%%%%%%%%   English Abstract  %%%%%%%%%%%%%%%%%%%%%%%%%%%%%%
\clearpage

\phantomsection
\markboth{Abstract}{Abstract}
\addcontentsline{toc}{chapter}{\xiaosi ABSTRACT}
\addcontentsline{toe}{chapter}{\bfseries \xiaosi Abstract (In English)}  \chapter*{\bf Abstract}
\@eabstract
\vspace{\baselineskip}

\hangafter=1\hangindent=60pt\noindent
{\textbf{Keywords:}}  \@ekeywords
}

%%%%%%%%%%%%%%%%%%%%%%%%%%%%%%%%%%%%%%%%%%%%%%%%%%%%%%%%%%%%%%%
% 英文目录格式
\def\@dotsep{0.75}           % 定义英文目录的点间距
\setlength\leftmargini {0pt}
\setlength\leftmarginii {0pt}
\setlength\leftmarginiii {0pt}
\setlength\leftmarginiv {0pt}
\setlength\leftmarginv {0pt}
\setlength\leftmarginvi {0pt}

\def\engcontentsname{\bfseries Contents}
\newcommand\tableofengcontents{
   \pdfbookmark[0]{Contents}{econtent}
     \@restonecolfalse
   \chapter*{\engcontentsname  %chapter*上移一行,避免在toc中出现。
       \@mkboth{%
          \engcontentsname}{\engcontentsname}}
   \@starttoc{toe}%
   \if@restonecol\twocolumn\fi
   }

\urlstyle{same}  %论文中引用的网址的字体默认与正文中字体不一致,这里修正为一致的。

\renewcommand\endtable{\vspace{-4mm}\end@float}

\makeatother
